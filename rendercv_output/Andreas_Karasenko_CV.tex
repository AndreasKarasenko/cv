\documentclass[10pt, a4paper]{article}

% Packages:
\usepackage[
        ignoreheadfoot, % set margins without considering header and footer
        top=2 cm, % seperation between body and page edge from the top
        bottom=2 cm, % seperation between body and page edge from the bottom
        left=2 cm, % seperation between body and page edge from the left
        right=2 cm, % seperation between body and page edge from the right
        footskip=1.0 cm, % seperation between body and footer
        % showframe % for debugging 
    ]{geometry} % for adjusting page geometry
\usepackage[explicit]{titlesec} % for customizing section titles
\usepackage{tabularx} % for making tables with fixed width columns
\usepackage{array} % tabularx requires this
\usepackage[dvipsnames]{xcolor} % for coloring text
\definecolor{primaryColor}{RGB}{0, 79, 144} % define primary color
\usepackage{enumitem} % for customizing lists
\usepackage{fontawesome5} % for using icons
\usepackage{amsmath} % for math
\usepackage[
    pdftitle={Andreas Karasenko's CV},
    pdfauthor={Andreas Karasenko},
    colorlinks=true,
    urlcolor=primaryColor
]{hyperref} % for links, metadata and bookmarks
\usepackage[pscoord]{eso-pic} % for floating text on the page
\usepackage{calc} % for calculating lengths
\usepackage{bookmark} % for bookmarks
\usepackage{lastpage} % for getting the total number of pages
\usepackage[default, type1]{sourcesanspro} % for using source sans 3 font
\usepackage{ifthen}

% Some settings:
\pagestyle{empty} % no header or footer
\setcounter{secnumdepth}{0} % no section numbering
\setlength{\parindent}{0pt} % no indentation
\setlength{\topskip}{0pt} % no top skip
\makeatletter
\let\ps@customFooterStyle\ps@plain % Copy the plain style to customFooterStyle
\patchcmd{\ps@customFooterStyle}{\thepage}{
    \color{gray}\textit{\small Andreas Karasenko - Seite \thepage{} von \pageref*{LastPage}}
}{}{} % replace number by desired string
\makeatother
\pagestyle{customFooterStyle}

\titleformat{\section}{
        % make the font size of the section title large and color it with the primary color
        \Large\color{primaryColor}
    }{
    }{
    }{
        % print bold title, give 0.15 cm space and draw a line of 0.8 pt thickness
        % from the end of the title to the end of the body
        \textbf{#1}\hspace{0.15cm}\titlerule[0.8pt]\hspace{-0.1cm}
    }[] % section title formatting

\titlespacing{\section}{
        % left space:
        0pt
    }{
        % top space:
        0.3 cm
    }{
        % bottom space:
        0.2 cm
    } % section title spacing

\newcolumntype{L}[1]{
    >{\raggedright\let\newline\\\arraybackslash\hspace{0pt}}p{#1}
} % left-aligned fixed width column type
\newcolumntype{R}[1]{
    >{\raggedleft\let\newline\\\arraybackslash\hspace{0pt}}p{#1}
} % right-aligned fixed width column type
\newcolumntype{K}[1]{
    >{\let\newline\\\arraybackslash\hspace{0pt}}X
} % justified flexible width column type
\setlength\tabcolsep{-1.5pt} % no space between columns
\newenvironment{highlights}{
        \begin{itemize}[
                topsep=0pt,
                parsep=0.10 cm,
                partopsep=0pt,
                itemsep=0pt,
                after=\vspace{-1\baselineskip},
                leftmargin=0.4 cm + 3pt
            ]
    }{
        \end{itemize}
    } % new environment for highlights

\newenvironment{header}{
        \setlength{\topsep}{0pt}\par\kern\topsep\centering\color{primaryColor}\linespread{1.5}
    }{
        \par\kern\topsep
    } % new environment for the header

\newcommand{\placelastupdatedtext}{% \placetextbox{<horizontal pos>}{<vertical pos>}{<stuff>}
  \AddToShipoutPictureFG*{% Add <stuff> to current page foreground
    \put(
        \LenToUnit{\paperwidth-2 cm-0.2 cm+0.05cm},
        \LenToUnit{\paperheight-1.0 cm}
    ){\vtop{{\null}\makebox[0pt][c]{
        \small\color{gray}\textit{Last updated in March 2024}\hspace{\widthof{Last updated in March 2024}}
    }}}%
  }%
}%

% save the original href command in a new command:
\let\hrefWithoutArrow\href
 % new command for external links:
\renewcommand{\href}[2]{\hrefWithoutArrow{#1}{\mbox{\ifthenelse{\equal{#2}{}}{ }{#2 }\raisebox{.15ex}{\footnotesize \faExternalLink*}}}}

\let\originalTabularx\tabularx
\let\originalEndTabularx\endtabularx

\renewenvironment{tabularx}{\bgroup\centering\originalTabularx}{\originalEndTabularx\par\egroup}

% For TextEntrys (see https://tex.stackexchange.com/a/600/287984):
\def\changemargin#1#2{\list{}{\rightmargin#2\leftmargin#1\topsep=0pt\itemsep=0pt\parsep=0pt\parskip=0pt\labelwidth=0pt\itemindent=0pt\labelsep=0pt}\item[]}
\let\endchangemargin=\endlist 

% Ensure that generate pdf is machine readable/ATS parsable
\pdfgentounicode=1

\begin{document}
    \begin{header}
        \fontsize{30 pt}{30 pt}
        \textbf{Andreas Karasenko}

        \vspace{0.3 cm}

        \normalsize
        \mbox{\hrefWithoutArrow{tel:+4917687742456}{{\footnotesize\faPhone*}\hspace*{0.13cm}+49 176 87742456}}
        \hspace*{0.5 cm}
        \mbox{\hrefWithoutArrow{mailto:andreas.karasenko@googlemail.com}{{\small\faEnvelope[regular]}\hspace*{0.13cm}andreas.karasenko@googlemail.com}}
        \hspace*{0.5 cm}
        \mbox{{\small\faMapMarker*}\hspace*{0.13cm}Bayreuth}
        \hspace*{0.5 cm}
        \mbox{\hrefWithoutArrow{https://linkedin.com/in/andreas-karasenko-57163a212/}{{\small\faLinkedinIn}\hspace*{0.13cm}andreas-karasenko-57163a212/}}
        \hspace*{0.5 cm}
        \mbox{\hrefWithoutArrow{https://github.com/AndreasKarasenko}{{\small\faGithub}\hspace*{0.13cm}AndreasKarasenko}}
        \hspace*{0.5 cm}
    \end{header}

    \vspace{0.3 cm}


    \section{Ausbildung}

        \begin{tabularx}{
            \textwidth-0.4 cm-0.13cm
        }{
            L{0.85cm}
            K{0.2 cm}
            R{4.1 cm}
        }
            \textbf{PhD}
            &
            \textbf{Universität Bayreuth}, Betriebswirtschaftslehre

            \vspace{0.10 cm}

            &
            

            Okt. 2021 bis Nov. 2025
        \end{tabularx}

        \vspace{0.2 cm}
        \begin{tabularx}{
            \textwidth-0.4 cm-0.13cm
        }{
            L{0.85cm}
            K{0.2 cm}
            R{4.1 cm}
        }
            \textbf{MS}
            &
            \textbf{Universität Bayreuth}, Betriebswirtschaftslehre

            \vspace{0.10 cm}

            \begin{highlights}
                \item \textbf{Coursework:} Operations Research, Datenbanken und Informationssysteme, Deep Learning, Data Mining im Marketing
            \end{highlights}
            &
            

            Okt. 2018 bis Sept. 2021
        \end{tabularx}

        \vspace{0.2 cm}
        \begin{tabularx}{
            \textwidth-0.4 cm-0.13cm
        }{
            L{0.85cm}
            K{0.2 cm}
            R{4.1 cm}
        }
            \textbf{BS}
            &
            \textbf{Universität Bayreuth}, Betriebswirtschaftslehre

            \vspace{0.10 cm}

            &
            

            Okt. 2014 bis Sept. 2018
        \end{tabularx}


    
    \section{Arbeitserfahrung}

        \begin{tabularx}{
            \textwidth-0.4 cm-0.13cm
        }{
            K{0.2 cm}
            R{4.1 cm}
        }
            \textbf{Lehrstuhl für Marketing und Innovation}, Wissenschaftlicher Mitarbeiter

            \vspace{0.10 cm}

            \begin{highlights}
                \item Durchführung von Industrieprojekten an der Schnittstelle zwischen Marketing, KI und Programmierung.
                \item Anfertigen von wissenschaftlichen Artikeln und Konferenzbeiträgen.
                \item Betreuen von Abschlussarbeiten.
            \end{highlights}
            &
            Bayreuth, Deutschland

            Okt. 2021 bis Heute
        \end{tabularx}

        \vspace{0.2 cm}
        \begin{tabularx}{
            \textwidth-0.4 cm-0.13cm
        }{
            K{0.2 cm}
            R{4.1 cm}
        }
            \textbf{Lehrstuhl für Serious Games}, Hilfswissenschaftler

            \vspace{0.10 cm}

            \begin{highlights}
                \item Korrektur von Übungen und Klausuren
                \item Erstellen von Bash/Python Skripten zur Automatisierung von Punktevergabe und Plagiatserkennung
            \end{highlights}
            &
            Bayreuth, Deutschland

            Okt. 2019 bis Sept. 2021
        \end{tabularx}


    
    \section{Industrieprojekte}

        \begin{tabularx}{
            \textwidth-0.4 cm-0.13cm
        }{
            K{0.2 cm}
            R{4.1 cm}
        }
            \textbf{BioTexFuture (Adidas)}

            \vspace{0.10 cm}

            \begin{highlights}
                \item Koordination und Projektmanagement von Qualitativen / Quantitativen Analysen im Bereich von Konsumentenverhalten und -präferenzen.
                \item Fullstack-Entwicklung einer Kollaborativen Webseite als zentrale Sammlung aller Studien und Experimente.
                \item Integration in ein Content Management System  und Planung eines Chat-Bots.
                \item Verwendete Technologien: Django, Wagtail, PostgreSQL, Docker, Git, CI/CD.
            \end{highlights}
            &
            

            Okt. 2021 bis Nov. 2025
        \end{tabularx}


        \vspace{0.2 cm}
        \begin{tabularx}{
            \textwidth-0.4 cm-0.13cm
        }{
            K{0.2 cm}
            R{4.1 cm}
        }
            \textbf{Process Mining im E-Commerce (BF/M-Bayreuth)}

            \vspace{0.10 cm}

            \begin{highlights}
                \item Planung und Durchführung von Forschungsprojekten zum Thema Process-Mining im E-Commerce.
                \item Planung und Durchführung themenspezifischer Veranstaltungen.
                \item Identifizierung von Nutzungspotentialen und praktische Anwendung auf Customer Journey Daten (Server Logs der Produktwebseite).
                \item Verwendete Technologien: BupaR, PM4PY.
            \end{highlights}
            &
            

            Apr. 2023 bis Feb. 2024
        \end{tabularx}


        \vspace{0.2 cm}
        \begin{tabularx}{
            \textwidth-0.4 cm-0.13cm
        }{
            K{0.2 cm}
            R{4.1 cm}
        }
            \textbf{Marketing Intelligence für KMU (ESF-VHB)}

            \vspace{0.10 cm}

            \begin{highlights}
                \item Identifizierung von Nutzungspotentialen für Kundenkarten(daten) in Baumärkten.
                \item Durchführung von RFM-/ und Co-Occurence Analysen.
                \item Entwicklung und offline Evaluation eines Empfehlungssystems.
                \item Verwendete Technologien: R, Tensorflow, Keras, Pandas, Matplotlib.
            \end{highlights}
            &
            

            Okt. 2021 bis Dez. 2022
        \end{tabularx}



    
    \section{Paperprojekte}

        \begin{tabularx}{
            \textwidth-0.4 cm-0.13cm
        }{
            K{0.2 cm}
            R{4.1 cm}
        }
            \textbf{Sentiment in Marketing (Karasenko)}

            \vspace{0.10 cm}

            \begin{highlights}
                \item Method Paper zu Sentiment-Analyse im Marketing (Anwendungen, Modellierung, Evaluation, Interpretation).
                \item Erstellen von exemplarischen Quellcodes für alle notwendigen Schritte.
                \item To be submitted (Marketing ZFP)
            \end{highlights}
            &
            

            2024
        \end{tabularx}


        \vspace{0.2 cm}
        \begin{tabularx}{
            \textwidth-0.4 cm-0.13cm
        }{
            K{0.2 cm}
            R{4.1 cm}
        }
            \textbf{Measuring technology acceptance over time by online customer reviews based transfer learning (Baier, Karasenko, Rese)}

            \vspace{0.10 cm}

            \begin{highlights}
                \item Vergleich zwischen transfer learning und einer herkömmlichen Umfrage mit einem Strukturgleichungsmodel.
                \item Proof of Concept für den automatisierten Einsatz von KI zur Technologieakzeptanzmessung.
                \item To be submitted (Journal of Retailing and Consumer Services)
            \end{highlights}
            &
            

            2024
        \end{tabularx}


        \vspace{0.2 cm}
        \begin{tabularx}{
            \textwidth-0.4 cm-0.13cm
        }{
            K{0.2 cm}
            R{4.1 cm}
        }
            \textbf{Beyond Sentiment (Karasenko)}

            \vspace{0.10 cm}

            \begin{highlights}
                \item Empirische Evaluation zum Einsatz von KI für die Technologieakzeptanzmessung auf Basis von Online-Reviews.
                \item Vergleich von 18 Natural Language Processing Modellen. Darunter BERT, GPT-3, RoBERTa, CNN, RF.
                \item Im Review (Electronic Commerce Research and Applications).
            \end{highlights}
            &
            

            Dez. 2023
        \end{tabularx}



    
    \section{Zusätzliche Erfahrungen}

        \begingroup\leftskip=0.2 cm
        \advance\csname @rightskip\endcsname 0.2 cm
        \advance\rightskip 0.2 cm

        \textbf{Bayreuth AI Association (Mai 2023 - Heute):} Bi-Weekly Meetings in denen aktuelle Entwicklungen in KI, Tools und Libraries vorgestellt, diskutiert und in kurzen Demos präsentiert werden.
        \par\endgroup


    
    \section{Technologien}

        \begingroup\leftskip=0.2 cm
        \advance\csname @rightskip\endcsname 0.2 cm
        \advance\rightskip 0.2 cm

        \textbf{Programmiersprachen:} Python, R, Javascript/HTML/CSS, Bash
        \par\endgroup

        \vspace{0.2 cm}
        \begingroup\leftskip=0.2 cm
        \advance\csname @rightskip\endcsname 0.2 cm
        \advance\rightskip 0.2 cm

        \textbf{Frameworks und Libraries:} TensorFlow, Keras, Huggingface, Pandas, NumPy, Matplotlib, Scikit-learn, Django, Wagtail
        \par\endgroup

        \vspace{0.2 cm}
        \begingroup\leftskip=0.2 cm
        \advance\csname @rightskip\endcsname 0.2 cm
        \advance\rightskip 0.2 cm

        \textbf{Werkzeuge:} Git, Docker, Linux, GitHub Actions
        \par\endgroup


    

\end{document}